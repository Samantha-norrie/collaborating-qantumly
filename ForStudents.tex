\documentclass{article}

% Language setting
% Replace `english' with e.g. `spanish' to change the document language
\usepackage[english]{babel}

% Set page size and margins
% Replace `letterpaper' with `a4paper' for UK/EU standard size
\usepackage[letterpaper,top=2cm,bottom=2cm,left=3cm,right=3cm,marginparwidth=1.75cm]{geometry}

% Useful packages
\usepackage{amsmath}
\usepackage{graphicx}
\usepackage[colorlinks=true, allcolors=blue]{hyperref}

\title{Collaborating Quantumly: An Interactive Workshop on the Foundations of Quantum Computing (Instructor's notes)}

\begin{document}
\maketitle

Get ready to make your students Quantum Curious! This workshop introduces students to foundational Quantum Computing tools through collaborative active learning.
\\
Over the course of the workshop, you will
\begin{itemize}
    \item become comfortable with using basic Quantum Computing jargon
    \item learn how to design basic quantum systems that display quantum behaviour
    \item learn the basic mathematics behind the magic of Quantum Computing
\end{itemize}

\newline \newline

More specifically, the topics that this workshop will cover are
\begin{itemize}
    \item qubits
    \item superposition
    \item entanglement
    \item quantum computer hardware limitations
    \item Dirac notation
    \item Matrix notation
    \item Foundational Linear Algebra
\end{itemize}


\section{Activity 0: Find Your Partner}

Your quantum state piece is \textbf{$1|0\rangle$}. Use Born's rule to help you find your partner.

\section{Activity 1: Hopping Through Hoops and Clouds}

After receiving the material needed for this activity from your instructor

\begin{enumerate}
    \item Create the quantum system in your designated application (either \href{https://quantumcurious.org/misty-states/}{Misty States} or \href{https://quantum.ibm.com/composer/files/new}{IBM Composer})
    \item Describe the quantum system to your partner and have them create it in their application
    \item Switch applications and repeat these steps for another quantum system
\end{enumerate}

What were some of the difficulties that you and your partner faced? What are the similarities between Misty States and IBM Composer? Share your findings with your classmates. 
\section{Activity 2: But How Does This Really Work?}

After receiving instructions from your instructor, take the quantum systems that you have created in IBM Composer and plug them into QNotation. Feel free to create some new quantum systems by modifying the Qiskit code.

\section{Activity 3: Exploring Space with a Quantum Computer}

Find another pair of students in your class and begin playing Entanglion.

Are any of the topics that you encountered while playing the game familiar to you? How so?


\end{document}